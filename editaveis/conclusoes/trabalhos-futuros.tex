\section{Trabalhos Futuros}

Para os próximos passos no projeto, foram planejadas as seguintes atividades:

\begin{itemize}
    \item \textbf{Requisitos}: serão priorizados e desenvolvidos os requisitos elicitados assim como serão elicitados novos requisitos caso perceba-se a necessidade;
    \item \textbf{Métricas de Código-fonte}: novas métricas para garantir a qualidade serão adicionadas. Duplicações encontradas no código atual serão removidas;
    \item \textbf{Integração e Deploy Contínuo}: novas políticas de qualidade de código serão adicionadas à integração com o intuito evitar o aceite de código de baixa qualidade no repositório. Soluções para o problema encontrado com o deploy contínuo serão estudadas e, se possível, implementadas;
    \item \textbf{Testes de usabilidade}: serão realizados testes de usabilidade com um grupo controlado de usuários, através do deploy contínuo, para garantir que o processo do aplicativo está funcional;
    \item \textbf{Protótipos de alta fidelidade}: também para a realização de testes de usabilidade, porém com o intuito de validar se o processo do aplicativo está fácil de ser realizado e intuitivo;
    \item \textbf{Definir estratégias de marketing}: para poder garantir a integridade dos dados que serão atualizados pelos próprios usuários, é de suma importância que o aplicativo tenha visibilidade;
    \item \textbf{Testes de desempenho}: para estudar melhor a escalabilidade do aplicativo, serão feitos testes de desempenho como testes de carga, testes de stress e testes de volume;
\end{itemize}
