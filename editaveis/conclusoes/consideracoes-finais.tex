\section{Considerações Finais}
Com o fechamento do trabalho tem-se um aplicativo móvel com impacto social, com intuito de melhorar o controle sobre postos de combustíveis por parte da sociedade. Conceitos de Engenharia de Software como, métricas de código, desenvolvimento ágil, testes de software automatizados, prototipação, integração e entrega contínua foram utilizadas para chegar ao resultado final.

Fazer a aplicação utilizando o Ionic 2 foi um grande risco. Infelizmente, devido aos problemas nas atualizações do Ionic e os plugins utilizados, a maior vantagem de se fazer uma aplicação em Ionic ao invés de nativo não foi utilizada, que é ter aplicativos para as duas principais plataformas com um só código. Aplicar conceitos de engenharia de software em um \textit{framework} relativamente novo ainda crescendo uma comunidade em torno dele foi uma experiência de grande valia, porém a conclusão que se teve foi que para projetos de grande porte o desenvolvimento nativo ainda é mais seguro.

O Ionic e seus criadores têm demostrado que a plataforma ainda irá crescer muito e espera-se que em suas versões subsequentes à versão 2, problemas com dependências e com o ambiente de desenvolvimento sejam resolvidos fazendo com que a plataforma seja mais estável. Para o desenvolvimento da API, não houve grandes problemas. O Ruby é uma linguagem muito estável e todos os pequenos problemas encontrados foram facilmente resolvidos devido a extensa documentação e comunidade.

O módulo de administração se mostrou muito útil inclusive nas fases de desenvolvimento, garantindo que os dados fossem efetivamente salvos e facilitando a criação de dados fictícios para testes. Na continuação deste projeto, certamente serão mantidos o módulo de administração assim como a API, e estudada a possibilidade de migração para a versão mais recente do Ionic ou para desenvolvimento nativo.
