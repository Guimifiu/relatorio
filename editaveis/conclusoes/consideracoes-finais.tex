\section{Considerações Finais}
Fazer a aplicação em Ionic 2 foi um grande risco. Infelizmente, devido aos problemas nas atualizações do Ionic e os plugins utilizados, a maior vantagem de se fazer uma aplicação em Ionic ao invés de nativo não foi utilizada. Desenvolvimento nativo, além de mais estável, iria trazer mais confiança no uso de plugins, principalmente de geolocalização e \textit{push notification}. Se pudessemos recomeçar o projeto, certamente utilizariamos desenvolvimento nativo. Para a API, não houveram grandes problemas. O Ruby é uma linguagem muito estável e todos os pequenos problemas encontrados foram facilmente resolvidos devido a extensa documentação e comunidade. 

O módulo de administração se mostrou muito útil inclusive nas fases de desenvolvimento, garantindo que os nossos dados estavam sendo efetivamente salvos e facilitando a criação de dados fictícios para testes. Na continuação deste projeto, certamente serão mantidos o módulo de administração e a API.
