\section{Contexto}

Em 2013, Brasil era o 39º país com a gasolina mais cara no mundo \cite{oglobo}. Ao analisar o \textit{ranking} dos preços de gasolina, é possível observar que os países que exportam gasolina possuem os menores preços, porém isso não acontece no Brasil \cite{ranking-precos}. Em 2015, a Polícia Federal e o Ministério Público do DF descobriram um esquema de cartel que combinava os valores dos combustíveis nos postos; mesmo após essa descoberta, os preços não diminuíram \cite{correio-cartel}.

Para mudar esse quadro, observa-se uma necessidade de um movimento comunitário da população brasileira para desestimular a existência dos cartéis. As manifestações de 2013 foram um grande exemplo da nação unida por uma mudança fazendo efeito, uma vez que as tarifas de ônibus diminuíram devido a movimentação popular. A falta de mecanismos de divulgação de suspeitas de cartéis \cite{manifestacoes-2013} e adulteração de combustíveis, entre outras situações fazem com que estas não tenham repercussão negativa para os donos dos postos. Foram tentados boicotes aos postos na época em que se descobriu o esquema de cartel, mas, pela falta de comunicação eficaz e organização, o resultado almejado não foi obtido \cite{boicotes-2016}.

Diante dessa realidade, foi concebido o Guimifiu, um aplicativo de celular que incentiva a população a boicotar postos de combustíveis cujos preços sejam muito al- tos. O aplicativo gerará uma lista de postos e os postos boicotados, onde o usuário será incentivado a não abastecer o veículo. Além disso, outros postos com preços menores e melhor qualidade serão recomendados, como forma de incentivar a diminuição dos preços. As informações de preços sobre os postos serão mantidas por meio de informações providas pelos usuários de forma colaborativa. O usuário também receberá uma recomendação de postos próximos a uma rota traçada por ele, pois não necessariamente o posto com melhores preços e melhor qualidade é o mais conveniente. O aplicativo foi feito utilizando conceitos e métodos aprendidos na engenharia de software, como o \textit{Scrum}, integração contínua e gerência de qualidade.
