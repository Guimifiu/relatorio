\section{Contexto}

O Brasil em 2013 era 39º país com a gasolina mais cara no mundo \cite{oglobo}. Ao analisar o ranking dos preços de gasolina é possível observar que os países que exportam gasolina são onde os preços são menores, mas infelizmente isso não acontece no Brasil \cite{ranking-precos}. Em 2015, a Polícia Federal e o Ministério Público do DF descobriram um esquema de cartel que combinava os valores de combustíveis nos postos, e mesmo após esta descoberta, os preços não diminuiram \cite{correio-cartel}.

Para mudar esse quadro, observa-se uma necessidade de um movimento comunitário da população brasileira para desistimular a existência dos cartéis. As manifestações de 2013 foram um grande exemplo da nação unida por uma mudança fazendo efeito. A falta de mecanismos para as pessoas divulgarem suspeitas de cartéis \cite{manifestacoes-2013}, adulteração de combustíveis, entre outras situações fazem com que essas situações aconteçam sem nenhuma repercussão negativa para os donos dos postos. Foram tentados boicotes aos postos na época em que se descobriu o esquema de cartel, mas pela falta de comunicação eficaz e organização, não obtiveram resultados positivos \cite{boicotes-2016}.

Diante dessa realidade, foi concebido o Guimifiu, um aplicativo de celular que incentiva a população a boicotar postos de combustíveis cujos preços sejam muito altos. O aplicativo gerará uma lista de postos, os postos boicotados, onde o usuário será incentivado a não abastecer o carro. Além disso, outros postos com preços menores e qualidade maior serão recomendados como forma de incentivar a diminuição dos preços. As informações de preços sobre os postos serão mantidas por meio de informações providas pelos usuários de forma colaborativa. O usuário também terá uma recomendação de postos próximos a uma rota traçada por ele, pois não necessariamente o posto com melhores preços e maior qualidade é o que mais vale a pena abastecer. O aplicativo foi feito utilizando conceitos e métodos aprendidos na engenharia de software, como o Scrum, integração contínua e gerência de qualidade.
