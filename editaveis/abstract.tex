\begin{resumo}[Abstract]
 \begin{otherlanguage*}{english}

     Among the oil-exporter countries, Brazil has one of the world's most expensive gas prices. In 2015, the existance of gas station cartels was uncovered, but even after the discovery and attempted boycott by the population, prices remained high. Guimifiu is a cross-platform application implemented with the aim of encouraging the Brazilian population to protest against abusive fuel prices in the country. Programmed in Ionic 2 with an API in Ruby on Rails, the application is being developed using methods and concepts of software engineering, such as Scrum and continuous delivery. Ionic 2's updates have prevented the work from achieving desired results. Nonetheless, the final version of the application is functional and ready to go into production and be evolved, mainly because the API and the administration module are well developed.
     \vspace{\onelineskip}
        
     \noindent
     \textbf{Keywords}: Application. Cross-platform. Boycott. Gas stations. Software Engineering. 
 \end{otherlanguage*}
\end{resumo}
