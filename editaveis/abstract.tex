\begin{resumo}[Abstract]
 \begin{otherlanguage*}{english}

     Brazil has one of the world's most expensive gas prices among the oil exporter countries. By 2015, cartels were uncovered at the gas stations, but even after the discovery and attempted boycott by the population, prices remained high. Guimifiu is a cross-platform application implemented with the aim of encouraging the Brazilian population to protest against abusive fuel prices in the country. Developed in Ionic 2 with an API in Ruby on Rails, the application is being developed using methods and concepts of software engineering, such as Scrum and continuous delivery. Despite the difficulties encountered with Ionic 2, the basis of the work is well prepared for the execution of the process on the second delivery.

     \vspace{\onelineskip}
        
     \noindent
     \textbf{Palavras-chaves}: application. cross-platform. boycott. gas stations. software engineering. 
 \end{otherlanguage*}
\end{resumo}
