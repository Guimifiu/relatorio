\begin{resumo}[Abstract]
 \begin{otherlanguage*}{english}

     Brazil has one of the world's most expensive gas prices among the oil exporter countries. By 2015, cartels were uncovered at the gas stations, but even after the discovery and attempted boycott by the population, prices remained high. Guimifiu is a cross-platform application implemented with the aim of encouraging the Brazilian population to protest against abusive fuel prices in the country. Programmed in Ionic 2 with an API in Ruby on Rails, the application is being developed using methods and concepts of software engineering, such as Scrum and continuous delivery. Ionic 2's updates have prevented the work from achieving desired results. Despite this, the final delivery of the application is functional and ready to go into production and be evolved, mainly because of the API and the administration module being well developed.
     \vspace{\onelineskip}
        
     \noindent
     \textbf{Keywords}: Application. Cross-platform. Boycott. Gas stations. Software Engineering. 
 \end{otherlanguage*}
\end{resumo}
