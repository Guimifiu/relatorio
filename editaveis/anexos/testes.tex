\chapter{Testes no Ionic 2}
\label{chap:testes}

\definecolor{lightgray}{rgb}{.9,.9,.9}
\definecolor{darkgray}{rgb}{.4,.4,.4}
\definecolor{purple}{rgb}{0.65, 0.12, 0.82}

\lstdefinelanguage{JavaScript}{
  keywords={typeof, new, true, false, catch, function, return, null, catch, switch, var, if, in, while, do, else, case, break},
  keywordstyle=\color{blue}\bfseries,
  ndkeywords={class, export, boolean, throw, implements, import, this},
  ndkeywordstyle=\color{darkgray}\bfseries,
  identifierstyle=\color{black},
  sensitive=false,
  comment=[l]{//},
  morecomment=[s]{/*}{*/},
  commentstyle=\color{purple}\ttfamily,
  stringstyle=\color{red}\ttfamily,
  morestring=[b]',
  morestring=[b]"
}

\lstset{
   language=JavaScript,
   backgroundcolor=\color{lightgray},
   extendedchars=true,
   basicstyle=\footnotesize\ttfamily,
   showstringspaces=false,
   showspaces=false,
   numbers=left,
   numberstyle=\footnotesize,
   numbersep=9pt,
   tabsize=2,
   breaklines=true,
   showtabs=false,
   captionpos=b
}

O teste unitário relizado em serviços precisa de um \textit{mock} para o \textit{backend}, uma vez que o objetivo é testar a unidade separadamente.
O código abaixo é um exemplo de teste unitário de um serviço simulando a comunicação com a API, para garantir que o serviço funciona por si só.
É criado um usuário fictício e quando o serviço é chamado para procurar um usuário pelo email, o usuário retornado tem que ter o nome e e sobrenome definido no \textit{mock}.

\medskip
\begin{lstlisting}
import { TestBed, inject, async } from '@angular/core/testing';
import { Http, HttpModule, BaseRequestOptions, Response, ResponseOptions } from '@angular/http';
import { MockBackend } from '@angular/http/testing';

import { User } from '../../../models/user';
import { UserService } from '../../../providers/user-service';

describe('Unit test: Providers: UserService', () => {
  beforeEach(async(() => {
    TestBed.configureTestingModule({
      declarations: [],
      providers: [
        UserService,
        MockBackend,
        BaseRequestOptions,
        {
          provide: Http,
          useFactory: (mockBackend, options) => {
            return new Http(mockBackend, options);
          },
          deps: [MockBackend, BaseRequestOptions]
        }
      ],
    imports: [ HttpModule ]
    }).compileComponents();
  }));

  describe('getUser()', () => {
    it('should get an user', inject([UserService, MockBackend], (userService, mockBackend) => {
          let user = new User();
          const mockResponse = {
            "id": 1,
            "email": "apptest@apptest.com",
            "created_at": "2017-04-19T01:06:15.145Z",
            "updated_at": "2017-04-19T01:06:15.145Z",
            "name": "App",
            "surname": "Teste",
            "document_number": null,
            "phone": null,
            "uid": null,
            "oauth_token": null,
            "provider": null
          }

          mockBackend.connections.subscribe((connection) => {
              connection.mockRespond(new Response(new ResponseOptions({
                  body: mockResponse
              })));
          });

          userService.getUser('apptest@apptest.com')
          .then(response => {
            user = response;
            expect(user.name).toBe('App');
            expect(user.surname).toBe('Teste');
          })
      }));
  });
});
\end{lstlisting}

Foi realizado o mesmo teste no serviço de usuários porém agora integrado com a API sem a criação de um \textit{mock} para ela.
O código abaixo mostra que antes de cada teste ser relizado um usuário é criado na API e depois de cada teste ele é deletado.
Ao pedir para a API retornar o usuário pelo email, o nome e o sobrenome tem que ser os mesmos do usuário criado antes do teste acontecer.


\medskip
\begin{lstlisting}
import {HttpModule} from '@angular/http';
import {async, TestBed,getTestBed } from '@angular/core/testing';

import { User } from '../../../models/user';
import { UserService } from '../../../providers/user-service';

let userService: UserService;
let user : User;

describe('Integration test: Providers: UserService', () => {
  beforeEach(async(() => {
   TestBed.configureTestingModule({
     imports: [HttpModule],
     providers:[UserService]
    })
    .compileComponents();
  }));

  beforeEach(async(() => {
    user = new User();
    user.name = "App";
    user.surname = "Teste";
    user.email = 'apptest@apptest.com';
    user.password = '12345678';
    user.provider = 'email';
    userService = getTestBed().get(UserService);
    userService.create(user);
  }));

  afterEach(async(() => {
    userService.delete(user);
  }));

  describe('getUser()', () => {
    it('should get User', async(() => {
      let user = new User();
      userService.getUser('apptest@apptest.com')
      .then(response => {
        user = response;
        expect(user.name).toBe('App');
        expect(user.surname).toBe('Teste');
      })
    }));
  });
});
\end{lstlisting}
