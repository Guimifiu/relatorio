\chapter{Dicionário de Dados}
\label{chap:dicionario}

A tabela \ref{dic:user} mostra o dicionário de dados da entidade \textit{User}, que irá armazenar todas as informações dos usuários que se cadastrarem no aplicativo.
\begin{table}[H]
\centering
\caption{Dicionário de dados da entidade \textit{User}}
\label{dic:user}
\begin{tabular}{|c|c|c|c|}\hline
\textbf{Atributo} & \textbf{Domínio} & \textbf{Tamanho} & \textbf{Descrição} \\
 \hline
Name                            & Texto                          & 45                             & Nome do usuário \\ \hline
Surname                         & Texto                          & 45                             & Sobrenome do usuário\\ \hline
Email                           & Texto                          & 45                             & Email do usuário\\ \hline
Password                        & Texto                          & 45                             & \begin{tabular}[l]{@{}c@{}}Senha do usuário. Deve possuir um mínimo \\ de 8 caracteres\end{tabular}\\ \hline

\end{tabular}
\end{table}

A tabela \ref{dic:rating} mostra o dicionário de dados da entidade \textit{Rating}, que irá armazenar as notas dadas pelos usuários para os postos de gasolina.
\begin{table}[H]
\centering
\caption{Dicionário de dados da entidade \textit{Rating}}
\label{dic:rating}
\begin{tabular}{|c|c|c|c|}\hline
\textbf{Atributo} & \textbf{Domínio} & \textbf{Formato} & \textbf{Descrição} \\ \hline

Stars                           & Númerico                       & 1                              & \begin{tabular}[l]{@{}c@{}}Classificação de um posto de gasolina.\\ Pode ir de 0 a 5\end{tabular}\\ \hline

\end{tabular}
\end{table}

A tabela \ref{dic:gas_station} mostra o dicionário de dados da entidade \textit{GasStation}, que irá armazenar todas as informações dos postos de gasolina importados da API do Google Maps além de informações adicionadas pelos usuários do Guimifiu.
\begin{table}[H]
\centering
\caption{Dicionário de dados da entidade \textit{GasStation}}
\label{dic:gas_station}
\begin{tabular}{|c|c|c|c|}\hline
\textbf{Atributo} & \textbf{Domínio} & \textbf{Tamanho} & \textbf{Descrição} \\ \hline
Latitude                        & Texto                          & 45                             & \begin{tabular}[l]{@{}c@{}}Latitude do posto de combustível.\\ Pode ir de 0 a 5\end{tabular}                 \\ \hline
Longitude                       & Texto                          & 45                             & Longitude do posto de combustível\\ \hline
Vicinity                        & Texto                          & 45                             & \begin{tabular}[l]{@{}c@{}}Endereço da vizinhança\\ onde o posto se localiza\end{tabular} \\ \hline
Google Maps Id                  & Númerico                       &                                & Chave da api do Google Maps      \\ \hline
Name                            & Texto                          & 45                             & Nome do posto de Gasolina\\ \hline
\end{tabular}
\end{table}

A tabela \ref{dic:pricesuggestion} mostra o dicionário de dados da entidade \textit{PriceSuggestion}, que irá armazenar informações sobre a sugestão de preço dos combustíveis em um posto, por usuários que já estiveram nele.
\begin{table}[H]
\centering
\caption{Dicionário de dados da entidade \textit{PriceSuggestion}}
\label{dic:pricesuggestion}
\begin{tabular}{|c|c|c|c|}\hline
\textbf{Atributo} & \textbf{Domínio} & \textbf{Formato} & \textbf{Descrição} \\ \hline
Type                            & Numérico                       & 1                              & \begin{tabular}[l]{@{}c@{}}Define o tipo de combustível\\(Gasolina, Álcool ou Diesel)\end{tabular}\\ \hline
Value                           & Numérico                       & 2,2                          & Valor do combustível\\ \hline

\end{tabular}
\end{table}

A tabela \ref{dic:fuelsupply} mostra o dicionário de dados da entidade \textit{FuelSupply}, que irá armazenar as informações sobre abastecimentos de usuários em posto de combustíveis.
\begin{table}[H]
\centering
\caption{Dicionário de dados da entidade \textit{FuelSupply}}
\label{dic:fuelsupply}
\begin{tabular}{|c|c|c|c|}\hline
\textbf{Atributo} & \textbf{Domínio} & \textbf{Formato} & \textbf{Descrição}\\ \hline
Date                            & Data                           & dd/mm/yyyy                       & Data que o usuário abasteceu no posto\\ \hline
Fueled                          & Boleano                        &                                & \begin{tabular}[l]{@{}c@{}}Define se o usuário abasteceu ou não \\no posto em que parou\end{tabular}\\ \hline
Boycotted                       & Boleano                        &                                & \begin{tabular}[l]{@{}c@{}}Define se o posto que o usuário parou\\ estava na lista de postos boicotados\end{tabular} \\ \hline
Value                           & Numérico                       & 2,2                             & Valor de combustível abastecido\\ \hline

\end{tabular}
\end{table}

A tabela \ref{dic:boycott} mostra o dicionário de dados da entidade \textit{Boycott}, que irá armazenar os dados dos eventos chamados de boicotes a postos de combustíveis, que aconteceram com uma peridiocidade definida.
\begin{table}[H]
\centering
\caption{Dicionário de dados da entidade \textit{Boycott}}
\label{dic:boycott}
\begin{tabular}{|c|c|c|c|}\hline
\textbf{Atributo} & \textbf{Domínio} & \textbf{Tamanho} & \textbf{Descrição}   \\ \hline
Start Date                      & Data                           & dd/mm/yyyy                & Data de início da lista de boicote \\ \hline
End Date                        & Data                           & dd/mm/yyyy                & Data final da lista de boicote\\ \hline

\end{tabular}
\end{table}
