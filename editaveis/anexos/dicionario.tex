\chapter{Dicionário de Dados}
\label{chap:dicionario}

A tabela \ref{dic:user} mostra o dicionário de dados da entidade \textit{User}, que irá armazenar todas as informações dos usuários que se cadastrarem no aplicativo.
\begin{table}[H]
\centering
\caption{Dicionário de dados da entidade \textit{User}}
\label{dic:user}
\begin{tabular}{cccc}
\toprule
\textbf{Atributo} & \textbf{Domínio} & \textbf{Tamanho} & \textbf{Descrição} \\
\midrule
Name                            & Texto                          & 45                             & Nome do usuário \\
Surname                         & Texto                          & 45                             & Sobrenome do usuário\\
Email                           & Texto                          & 45                             & Email do usuário\\
Password                        & Texto                          & 45                             & Senha do usuário. Deve possuir um mínimo de 8 caracteres\\
\bottomrule
\end{tabular}
\end{table}

A tabela \ref{dic:rating} mostra o dicionário de dados da entidade \textit{Rating}, que irá armazenar as notas dadas pelos usuários para os postos de gasolina.
\begin{table}[H]
\centering
\caption{Dicionário de dados da entidade \textit{Rating}}
\label{dic:rating}
\begin{tabular}{cccc}
\toprule
\textbf{Atributo} & \textbf{Domínio} & \textbf{Tamanho} & \textbf{Descrição} \\
\midrule
Stars                           & Númerico                       &1                               & Classificação de um posto de gasolina. Pode ir de 0 a 5\\
\bottomrule
\end{tabular}
\end{table}

A tabela \ref{dic:gas_station} mostra o dicionário de dados da entidade \textit{GasStation}, que irá armazenar todas as informações dos postos de gasolina importados da API do Google Maps além de informações adicionadas pelos usuários do Guimifiu.
\begin{table}[H]
\centering
\caption{Dicionário de dados da entidade \textit{GasStation}}
\label{dic:gas_station}
\begin{tabular}{cccc}
\toprule
\textbf{Atributo} & \textbf{Domínio} & \textbf{Tamanho} & \textbf{Descrição} \\
\midrule
Latitude                        & Texto                          & 45                             & Latitude do posto de combustível. Pode ir de 0 a 5                 \\
Longitude                       & Texto                          & 45                             & Longitude do posto de combustível\\
Vicinity                        & Texto                          & 45                             & Endereço da visinhança onde o posto se localiza \\
Google Maps Id                  & Númerico                       &                                & Chave da api do Google Maps      \\
Name                            & Texto                          & 45                             & Nome do posto de Gasolina\\
\bottomrule
\end{tabular}
\end{table}

A tabela \ref{dic:pricesuggestion} mostra o dicionário de dados da entidade \textit{PriceSuggestion}, que irá armazenar informações sobre a sugestão de preço dos combustíveis em um posto, por usuários que já estiveram nele.
\begin{table}[H]
\centering
\caption{Dicionário de dados da entidade \textit{PriceSuggestion}}
\label{dic:pricesuggestion}
\begin{tabular}{cccc}
\toprule
\textbf{Atributo} & \textbf{Domínio} & \textbf{Tamanho} & \textbf{Descrição} \\
\midrule
Type                            & Numérico                       &1                               & Define o tipo de combustível (Gasolina, Álcool ou Diesel)\\
Value                           & Numérico                       &2,2                             & Valor do combustível\\
\bottomrule
\end{tabular}
\end{table}

A tabela \ref{dic:fuelsupply} mostra o dicionário de dados da entidade \textit{FuelSupply}, que irá armazenar as informações sobre abastecimentos de usuários em posto de combustíveis.
\begin{table}[H]
\centering
\caption{Dicionário de dados da entidade \textit{FuelSupply}}
\label{dic:fuelsupply}
\begin{tabular}{cccc}
\toprule
\textbf{Atributo} & \textbf{Domínio} & \textbf{Tamanho} & \textbf{Descrição}\\
\midrule
Date                            & Data                           &dd/mm/yy                        & Data que o usuário abasteceu no posto\\
Fueled                          & Boleano                        &                                & \begin{tabular}[l]{@{}c@{}}Define se o usuário abasteceu ou não \\no posto em que parou\end{tabular}\\
Boycotted                       & Boleano                        &                                & \begin{tabular}[l]{@{}c@{}}Define se o posto que o usuário parou\\ estava na lista de postos boicotados\end{tabular} \\
Value                           & Numérico                       &2,2                             & Valor de combustível abastecido\\
\bottomrule
\end{tabular}
\end{table}

A tabela \ref{dic:boycott} mostra o dicionário de dados da entidade \textit{Boycott}, que irá armazenar os dados dos eventos chamados de boicotes a postos de combustíveis, que aconteceram com uma peridiocidade definida.
\begin{table}[H]
\centering
\caption{Dicionário de dados da entidade \textit{Boycott}}
\label{dic:boycott}
\begin{tabular}{cccc}
\toprule
\textbf{Atributo} & \textbf{Domínio} & \textbf{Tamanho} & \textbf{Descrição}   \\
\midrule
Start Date                      & Data                           &dd/mm/yy                         & Data de início da lista de boicote \\
End Date                        & Data                           &dd/mm/yy                         & Data final da lista de boicote\\
\bottomrule
\end{tabular}
\end{table}
