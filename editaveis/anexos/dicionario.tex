\chapter{Dicionário de Dados}
\label{chap:dicionario}

A Tabela \ref{dic:user} mostra o dicionário de dados da entidade \textit{Usuario}, que irá armazenar todas as informações dos usuários que se cadastrarem no aplicativo.
\begin{table}[H]
\centering
\caption{Dicionário de dados da entidade \textit{Usuario}}
\label{dic:user}
\begin{tabular}{|c|c|c|c|}\hline
\textbf{Atributo} & \textbf{Domínio} & \textbf{Descrição} \\
 \hline
Nome                            & Texto                                                       & Nome do usuário \\ \hline
Sobrenome                         & Texto                                                   & Sobrenome do usuário\\ \hline
Email                           & Texto                                              & Email do usuário\\ \hline
Senha                        & Texto                                             & \begin{tabular}[l]{@{}c@{}}Senha do usuário. Deve possuir um mínimo \\ de 8 caracteres\end{tabular}\\ \hline

\end{tabular}
\end{table}

A Tabela \ref{dic:rating} mostra o dicionário de dados da entidade \textit{Avaliacao}, que irá armazenar as notas dadas pelos usuários para os postos de gasolina.
\begin{table}[H]
\centering
\caption{Dicionário de dados da entidade \textit{Avaliacao}}
\label{dic:rating}
\begin{tabular}{|c|c|c|c|}\hline
\textbf{Atributo} & \textbf{Domínio} & \textbf{Descrição} \\ \hline
Estrelas                           & Númerico                                               & \begin{tabular}[l]{@{}c@{}}Classificação de um posto de gasolina.\\ Pode ir de 0 a 5\end{tabular}\\ \hline

\end{tabular}
\end{table}

A Tabela \ref{dic:gas_station} mostra o dicionário de dados da entidade \textit{PostoDeCombustivel}, que irá armazenar todas as informações dos postos de gasolina importados da API do Google Maps além de informações adicionadas pelos usuários do Guimifiu.
\begin{table}[H]
\centering
\caption{Dicionário de dados da entidade \textit{PostoDeCombustivel}}
\label{dic:gas_station}
\begin{tabular}{|c|c|c|c|}\hline
\textbf{Atributo} & \textbf{Domínio} & \textbf{Descrição} \\ \hline
Latitude                        & Texto                                                     & \begin{tabular}[l]{@{}c@{}}Latitude do posto de combustível.\\ Pode ir de 0 a 5\end{tabular}                 \\ \hline
Longitude                       & Texto                                                       & Longitude do posto de combustível\\ \hline
Vizinhanca                        & Texto                                                     & \begin{tabular}[l]{@{}c@{}}Endereço da vizinhança\\ onde o posto se localiza\end{tabular} \\ \hline
Google Maps Id                  & Númerico                                                       & Chave da api do Google Maps      \\ \hline
Nome                            & Texto                                                      & Nome do posto de Gasolina\\ \hline
\end{tabular}
\end{table}

A Tabela \ref{dic:pricesuggestion} mostra o dicionário de dados da entidade \textit{SugestaoDePreco}, que irá armazenar informações sobre a sugestão de preço dos combustíveis em um posto, por usuários que já estiveram nele.
\begin{table}[H]
\centering
\caption{Dicionário de dados da entidade \textit{SugestaoDePreco}}
\label{dic:pricesuggestion}
\begin{tabular}{|c|c|c|c|}\hline
\textbf{Atributo} & \textbf{Domínio} & \textbf{Descrição} \\ \hline
Tipo                            & Numérico                                                     & \begin{tabular}[l]{@{}c@{}}Define o tipo de combustível\\(Gasolina, Álcool ou Diesel)\end{tabular}\\ \hline
Valor                           & Numérico                                                 & Valor do combustível\\ \hline

\end{tabular}
\end{table}

A Tabela \ref{dic:fuelsupply} mostra o dicionário de dados da entidade \textit{Abastecimento}, que irá armazenar as informações sobre abastecimentos de usuários em posto de combustíveis.
\begin{table}[H]
\centering
\caption{Dicionário de dados da entidade \textit{Abastecimento}}
\label{dic:fuelsupply}
\begin{tabular}{|c|c|c|c|}\hline
\textbf{Atributo} & \textbf{Domínio} & \textbf{Descrição}\\ \hline
Data                            & Data                                                & Data que o usuário abasteceu no posto\\ \hline
Abastecido                          & Boleano                                           & \begin{tabular}[l]{@{}c@{}}Define se o usuário abasteceu ou não \\no posto em que parou\end{tabular}\\ \hline
Boicotado                       & Boleano                                                  & \begin{tabular}[l]{@{}c@{}}Define se o posto que o usuário parou\\ estava na lista de postos boicotados\end{tabular} \\ \hline
Valor                           & Numérico                                              & Valor de combustível abastecido\\ \hline

\end{tabular}
\end{table}

A Tabela \ref{dic:boycott} mostra o dicionário de dados da entidade \textit{Boicote}, que irá armazenar os dados dos eventos chamados de boicotes por lista comum a postos de combustíveis, que aconteceram com uma peridiocidade definida.

\begin{table}[H]
\centering
\caption{Dicionário de dados da entidade \textit{Boicote}}
\label{dic:boycott}
\begin{tabular}{|c|c|c|c|}\hline
\textbf{Atributo} & \textbf{Domínio}  & \textbf{Descrição}   \\ \hline
Data de Início                      & Data                                           & Data de início da lista de boicote \\ \hline
Data de Término                       & Data                                         & Data final da lista de boicote\\ \hline

\end{tabular}
\end{table}


Representado na Tabela \ref{dic:flag}, o dicionário de dados da entidade \textit{Bandeira} armazenará os dados das bandeiras dos postos de gasolina, auxiliando tanto na identificação quanto no boicote por bandeira.

\begin{table}[H]
\centering
\caption{Dicionário de dados da entidade \textit{Bandeira}}
\label{dic:flag}
\begin{tabular}{|c|c|c|c|}\hline
\textbf{Atributo} & \textbf{Domínio} & \textbf{Descrição}   \\ \hline
Nome                      & Texto                                           & Nome da bandeira \\ \hline
Caminho da Imagem                       & Texto                                           & Caminho da imagem da bandeira\\ \hline

\end{tabular}
\end{table}

Por fim, a Tabela \ref{dic:flag_boycott} representa o dicionário de dados da entidade \textit{BoicoteBandeira} que armazena os dados dos boicotes por bandeira.

\begin{table}[H]
\centering
\caption{Dicionário de dados da entidade \textit{BoicoteBandeira}}
\label{dic:flag_boycott}
\begin{tabular}{|c|c|c|c|}\hline
\textbf{Atributo} & \textbf{Domínio}  & \textbf{Descrição}   \\ \hline
Data de Início                      & Data                                           & Data de início da lista de boicote \\ \hline
Data de Término                       & Data                                         & Data final da lista de boicote\\ \hline

\end{tabular}
\end{table}
