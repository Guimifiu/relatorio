\section{Aplicação Desenvolvida}
Para a primeira entrega do aplicativo, foram desenvolvidos os seguintes requisitos:
\begin{itemize}
    \item Eu, como motorista, gostaria de me cadastrar no sistema para ter acesso ao login;
    \item Eu, como motorista, gostaria de logar no sistema para utilizar as funcionalidades do mesmo;
    \item Eu, como motorista, gostaria de ver os postos dentro de uma rota para facilitar a minha escolha de posto.
\end{itemize}
No Anexo \ref{chap:telas} encontram-se algumas telas da aplicação desenvolvida.

\subsection{Método de decisão de postos boicotados}
Toda semana, serão selecionados postos onde os usuários serão incentivados a não abastecer. Para decidir quais postos serão boicotados, o Guimifiu tem duas estratégias principais: a \textbf{Lista Comum} e o \textbf{Boicote por Bandeiras}. A lista comum escolhe os postos com pior qualidade geral e que não tenham sido boicotados recentemente e cria uma lista inicial. A qualidade geral de um posto é determinada pela média ponderada de estrelas do posto e pelo preço dos combustíveis considerando o preço de todos os postos. O preço tem peso 2 e a quantidade de estrelas tem peso 1. A lista inicial é montada e depois é verificado se não existem muitos postos perto demais sendo boicotados para não deixar o usuário com poucas opções de locais para abastecimento.

Essa verificação é feita através de grafos. Os nós dos grafos são os postos de combustíveis e as arestas são as distâncias entre os todos postos que não sejam maiores que 10 quilômetros. Com este grafo, em cada nó da lista inicial é verificado se os nós adjacentes são todos postos que estão na lista inicial. Em caso positivo, o nó é removido da lista, caso contrário o nó é mantido. Os nós restantes compõem a lista final de postos boicotados, denominada \textbf{Lista Comum}.

IMAGEM

IMAGEM

O \textit{Boicote por Bandeiras} escolhe uma bandeira e todos os postos dessa bandeira são colocados na lista de boicotados, independentemente da distância entre eles, colocando o prejuízo do boicote menos nos postos e mais na bandeira. Os dois tipos de boicote são exclusivos, ou seja, os dois não ocorrem no mesmo período. A estratégia de boicote é escolhida pela equipe de desenvolvimento.  

\subsection{Testes Automatizados}

Foram feitos testes de integração e unitários para garantir o funcionamento dos métodos do código fonte e da integração da aplicação \textit{mobile} com a API.

Foi encontrada uma certa dificuldade pra realizar os testes unitários no Ionic, uma vez que não há muitos exemplos que de fato funcionam e uma documentação do próprio \textit{framework} sobre o assunto.

O código de testes pode ser encontrado no repositório da aplicação no GitHub (https://github.com/Guimifiu/guimifiu-app). Um exemplo de teste unitário e um de teste de integração está disponível no Anexo \ref{chap:testes}.

\subsection{Métricas de Código fonte}

Como citado no \autoref{chap:met}, foram utilizadas as ferramentas CodeClimate e Rubocop. Além de medir a complexidade no Rubocop, foi feita na integração contínua um relatório de métricas pela ferramenta que impede que a \textit{build} passe caso a complexidade seja maior do que 6, que é o padrão de complexidade ciclomática máxima aceitável.

No Anexo \ref{chap:metricas} encontram-se os resultados da coleta de métricas do sistema.

\subsection{Integração e Entrega Contínua}
Há três tipos de ambientes no desenvolvimento desta aplicação: \textit{development}, \textit{staging} e \textit{production}. O ambiente de \textit{development}, ou desenvolvimento, é o ambiente local de cada desenvolvedor, \textit{staging} ou pré produção é um ambiente para serem realizados testes beta e \textit{production} ou produção é o ambiente final onde a aplicação será de fato utilizada pelos usuários.

Na API a integração contínua é feita com a ferramenta TravisCI. Onde, se um \textit{pull request} é aceito na \textit{branch} \textit{staging}, o Travis executa toda a suíte de testes para ver se algum deles pode ter resultado em falha. Caso todos passem, ele verifica se todas as métricas definidas pelo Rubocop estão de acordo com o padrão definido. Caso todas as métricas passem, a \textit{build} de pré produção é criada, o Travis envia as informações de cobertura de testes para o Coveralls, as informações de métricas para o Code Climate e logo em seguida realiza o \textit{deploy} no ambiente de pré produção no Heroku.

Caso um \textit{pull request} seja aceito na \textit{branch} \textit{master} todo o processo será o mesmo, com exceção que o \textit{deploy} irá ocorrer no ambiente de produção no Heroku.

A Figura \ref{img:integracao_deploy_continuo_api} ilustra o processo de integração e \textit{deploy} contínuo feito na API.

\begin{figure}[H]
    \centering
    \includegraphics[scale=0.5]{figuras/api_ci.png}
    \caption[Integração e \textit{deploy} contínuo API]{Integração e \textit{deploy} contínuo API. Fonte: autores}
    \label{img:integracao_deploy_continuo_api}
\end{figure}

No aplicativo foi planejada a integração e o \textit{deploy} contínuo para ocorrer como mostra a Figura \ref{img:integracao_deploy_continuo_planejado_app}.

\begin{figure}[H]
    \centering
    \includegraphics[scale=0.5]{figuras/ci_should_be.png}
    \caption[Integração e \textit{deploy} contínuo planejado para o aplicativo]{Integração e \textit{deploy} contínuo planejado para o aplicativo. Fonte: autores}
    \label{img:integracao_deploy_continuo_planejado_app}
\end{figure}

Se um \textit{pull request} é aceito na \textit{branch} \textit{master}, o Travis irá realizar toda a suíte de testes, caso todos passem enviará as informações de cobertura para o
Codecov e através do Fastlane realizará o \textit{deploy} do aplicativo tanto na Play Store quanto na App Store. E caso um \textit{pull request} seja aceito na \textit{branch} \textit{staging}, todo o processo é repetido, exceto quando não é realizado o \textit{deploy} do aplicativo nas lojas e sim no Crashlytics, que irá criar a versão beta do aplicativo e o enviará para os emails configurados previamente.

Uma vez que o aplicativo ainda não chegou ao estágio de \textit{deploy} nas lojas, o planejamento para a primeira entrega é só até o \textit{deploy} beta, como mostrado na Figura \ref{img:integracao_deploy_continuo_planejado_primeira_entrega}:

\begin{figure}[H]
    \centering
    \includegraphics[scale=0.5]{figuras/ci_as_is.png}
    \caption[Integração e \textit{deploy} contínuo planejado para a primeira entrega]{Integração e \textit{deploy} contínuo planejado para a primeira entrega. Fonte: autores}
    \label{img:integracao_deploy_continuo_planejado_primeira_entrega}
\end{figure}

Entretanto, não foi possível fazer com que o \textit{deploy} contínuo aconteça logo após a integração contínua, um desafio causado por escolher desenvolvimento híbridos de aplicativo. Uma vez que o código versionado do aplicativo é o mesmo código para a plataforma Android e iOS, e após realizar a \textit{build} desse código é gerado códigos em cada plataforma (não versionados). A ferramenta de integração contínua não consegue ter acesso aos códigos de cada plataforma para realizar os devidos \textit{deploys}. Sendo assim, o \textit{deploy} contínuo está acontecendo manualmente, por enquanto, até que solução seja encontrada, através do comando:

\begin{lstlisting}[language=bash]
  $ fastlane beta
\end{lstlisting}

Sendo assim, a Figura \ref{img:integracao_deploy_continuo_atual} ilustra como está funcionando a integração e o \textit{deploy} contínuo do aplicativo até o presente momento.

\begin{figure}[H]
    \centering
    \includegraphics[scale=0.5]{figuras/ci_currently.png}
    \caption[Integração e \textit{deploy} contínuo atual]{Integração e \textit{deploy} contínuo atual. Fonte: autores}
    \label{img:integracao_deploy_continuo_atual}
\end{figure}


