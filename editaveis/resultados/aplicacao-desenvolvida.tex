\section{Aplicação Desenvolvida}
Para a primeira entrega do aplicativo, foram desenvolvidos os seguintes requisitos:
\begin{itemize}
    \item Eu, como motorista, gostaria de me cadastrar no sistema para ter acesso ao login
    \item Eu, como motorista, gostaria de logar no sistema para utilizar as funcionalidades do mesmo
    \item Eu, como motorista, gostaria de ver os postos dentro de uma rota para facilitar a minha escolha de posto
\end{itemize}
No Anexo \ref{chap:telas} encontram-se algumas telas da aplicação desenvolvida.

\subsection{Testes Automatizados}

Foram feitos testes de integração e unitários para garantir o funcionamento dos métodos do código-fonte e da integração da aplicação mobile com a API.

No Anexo \ref{chap:testes} foi feito um tutorial de como realizar os testes unitários e funcionais no Ionic com um exemplo de um dos testes criados.

\subsection{Métricas de Código-fonte}

Como dito no \autoref{chap:met}, foram utilizadas as ferramentas CodeClimate e Rubocop. Além de medir a complexidade no Rubocop, é feita na integração contínua um relatório de métricas pela ferramenta que impede que a \textit{build} passe caso a complexidade seja maior do que 6, que é o padrão de complexidade ciclomática máxima aceitável.

No Anexo \ref{chap:metricas} é possível encontrar os resultados da coleta de métricas do sistema.

\subsection{Integração e Entrega Contínua}
Há três tipos de ambientes no desenvolvimento desta aplicação: \textit{development}, \textit{staging} e \textit{production}. O ambiente de \textit{development}, ou desenvolvimento, é o ambiente local de cada desenvolvedor, \textit{staging} ou pré produção é um ambiente para serem realizados testes beta e \textit{production} ou produção é o ambiente final onde a aplicação será de fato utilizada pelos usuários.

Na API a integração contínua é feita com a ferramenta TravisCI. Onde, se um \textit{pull request} é aceito na \textit{branch} \textit{staging}, o Travis executa toda a suíte de testes para ver se algum deles pode ter resultado em falha. Caso todos passem, ele verifica se todas as métricas definidas pelo Rubocop estão de acordo com o padrão definido. Caso todas as métricas passem, a \textit{build} de pré produção é criada, o Travis envia as informações de cobertura de testes para o Coveralls, as informações de métricas para o Code Climate e logo em seguida realiza o \textit{deploy} no ambiente de pré produção no Heroku.

Caso um \textit{pull request} seja aceito na \textit{branch} \textit{master} todo o processo será o mesmo, com exceção que o \textit{deploy} irá ocorrer no ambiente de produção no Heroku.

A Figura \ref{img:integracao_deploy_continuo_api} ilustra o processo de integração e \textit{deploy} contínuo feito na API.

\begin{figure}[H]
    \centering
    \includegraphics[scale=0.5]{figuras/api_ci.png}
    \caption[Integração e \textit{deploy} contínuo API]{Integração e \textit{deploy} contínuo API. Fonte: autores}
    \label{img:integracao_deploy_continuo_api}
\end{figure}

No aplicativo foi planejada a integração e o \textit{deploy} contínuo para ocorrer como mostra a Figura \ref{img:integracao_deploy_continuo_planejado_app}.

\begin{figure}[H]
    \centering
    \includegraphics[scale=0.5]{figuras/ci_should_be.png}
    \caption[Integração e \textit{deploy} contínuo planejado para o aplicativo]{Integração e \textit{deploy} contínuo planejado para o aplicativo. Fonte: autores}
    \label{img:integracao_deploy_continuo_planejado_app}
\end{figure}

Se um \textit{pull request} é aceito na \textit{branch} \textit{master}, o Travis irá realizar toda a suíte de testes, caso todos passem enviará as informações de cobertura para o
Codecov e através do Fastlane realizará o \textit{deploy} do aplicativo tanto na Play Store quanto na App Store. E caso um \textit{pull request} seja aceito na \textit{branch} \textit{staging}, todo o processo é repetido, exceto quando não é realizado o \textit{deploy} do aplicativo nas lojas e sim no Crashlytics, que irá criar a versão beta do aplicativo e o enviará para os emails configurados previamente.

Uma vez que o aplicativo ainda não chegou ao estágio de \textit{deploy} nas lojas, o planejamento para a primeira entrega é só até o \textit{deploy} beta, como mostrado na Figura \ref{img:integracao_deploy_continuo_planejado_primeira_entrega}:

\begin{figure}[H]
    \centering
    \includegraphics[scale=0.5]{figuras/ci_as_is.png}
    \caption[Integração e \textit{deploy} contínuo planejado para a primeira entrega]{Integração e \textit{deploy} contínuo planejado para a primeira entrega. Fonte: autores}
    \label{img:integracao_deploy_continuo_planejado_primeira_entrega}
\end{figure}

Entretanto, não foi possível fazer com que o \textit{deploy} contínuo aconteça logo após a integração contínua, um desafio causado por escolher desenvolvimento híbridos de aplicativo. Uma vez que o código versionado do aplicativo é o mesmo código para a plataforma Android e iOS, e após realizar a \textit{build} desse código é gerado códigos em cada plataforma (não versionados). A ferramenta de integração contínua não consegue ter acesso aos códigos de cada plataforma para realizar os devidos \textit{deploys}. Sendo assim, o \textit{deploy} contínuo está acontecendo manualmente, por enquanto, até que solução seja encontrada, através do comando:

\begin{lstlisting}[language=bash]
  $ fastlane beta
\end{lstlisting}

Sendo assim, a Figura \ref{img:integracao_deploy_continuo_atual} ilustra como está funcionando a integração e o \textit{deploy} contínuo do aplicativo até o presente momento.

\begin{figure}[H]
    \centering
    \includegraphics[scale=0.5]{figuras/ci_currently.png}
    \caption[Integração e \textit{deploy} contínuo atual]{Integração e \textit{deploy} contínuo atual. Fonte: autores}
    \label{img:integracao_deploy_continuo_atual}
\end{figure}

