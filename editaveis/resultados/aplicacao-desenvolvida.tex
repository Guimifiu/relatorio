\section{Aplicação Desenvolvida}
Para a primeira entrega do aplicativo, foram desenvolvidos os seguintes requisitos:
\begin{itemize}
    \item Eu, como motorista, gostaria de me cadastrar no sistema para ter acesso ao login 
    \item Eu, como motorista, gostaria de logar no sistema para utilizar as funcionalidades do mesmo
    \item Eu, como motorista, gostaria de ver os postos dentro de uma rota para facilitar a minha escolha de posto
\end{itemize}
No Anexo \ref{chap:telas} encontram-se algumas telas da aplicação desenvolvida.

\subsection{Testes Automatizados}

Foram feitos testes de integração e unitários para garantir o funcionamento dos métodos do código-fonte e da integração da aplicação mobile com a API.

Para a aplicação mobile, foram utilizados os \textit{frameworks open source} Jasmine para escrever os testes e o Karma para rodar a suíte. O Karma é um ambiente de teste para Javascript onde os desenvolvedores conseguem \textit{feedbacks} rápidos do código que estão desenvolvendo \cite{karma}. O Jasmine é um \textit{framework} que utiliza conceitos de \textit{behavior-driven development} para escrever testes \cite{jasmine}.

\subsection{Métricas de Código-fonte}

Para as métricas de código-fonte da aplicação mobile, foi utilizado o TSLint, que checa código TypeScript em questões de manutenabilidade, leitura e erros funcionais \cite{tslint}.

No código-fonte da API, foi utilizado o CodeClimate, uma ferramenta que baixa o código do Github e checa em seus servidores a complexidade do código, duplicação, estilo e segurança \cite{codeclimate}. O CodeClimate utiliza as seguintes para gerar o relatório final:
\begin{itemize}
    \item \textbf{Reek}, para verificação de \textit{code smells}, que são sintomas no código-fonte que podem gerar problemas;
    \item \textbf{Flog}, para a verificação da complexidade do código Ruby;
    \item \textbf{Rubocop}, que verifica estilo e qualidade do código;
    \item \textbf{Brakeman}, para uma verificação estática de segurança do código;
\end{itemize}

Além disso, é também uma ferramenta própria do CodeClimate para checar duplicações de código.

Adicionalmente, é feita na integração contínua um relatório de métricas pelo Rubocop que impede que a \textit{build} passe caso a complexidade seja maior do que 6, que é o padrão de complexidade ciclomática máxima aceitável. Além de complexidade, o Rubocop possui diversas outras ferramentas embutidas que verificam outros aspectos baseados no guia de estilo do Ruby mantido pela comunidade \cite{rubocop}. Os padrões mínimos ou máximos dessas outras ferramentas serão analisadas com o desenvolvimento do projeto.

\subsection{Integração Contínua}

\subsection{Deploy Contínuo}
