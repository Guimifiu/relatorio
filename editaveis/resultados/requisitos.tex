\section{Requisitos}

Para os requisitos, foram utilizadas histórias de usuário para escrever os itens de \textit{backlog} que serão desenvolvidos durante as \textit{sprints}. Os requisitos elicitados foram:
\begin{itemize}
    \item Eu, como motorista, gostaria de me cadastrar no sistema para ter acesso ao login
    \item Eu, como motorista, gostaria de logar no sistema para utilizar as funcionalidades do mesmo
    \item Eu, como motorista, gostaria de ver os postos dentro de uma rota para facilitar a minha escolha de posto
    \item Eu, como motorista, gostaria de avaliar os preços já existentes para colocar preços verdadeiros em cima e falsos em baixo
    \item Eu, como motorista, gostaria de avaliar o posto onde eu abasteci para atribuir uma qualidade ao mesmo
    \item Eu, como motorista, gostaria de informar se estou ou não abastecendo no posto para contribuir com os dados coletados pelo aplicativo
    \item Eu, como motorista, gostaria de ver uma lista de postos com melhor custo/benefício para abastecer no melhor posto para mim
    \item Eu, como motorista, gostaria de ver uma lista com todos os postos boicotados para me programar com antecedência onde abastecer
    \item Eu, como motorista, gostaria de saber quando eu estou em um posto de combustíveis para interagir com o mesmo
    \item Eu, como motorista, gostaria de ver as informações do posto onde eu estou, para saber se quero abastecer nele ou não
    \item Eu, como motorista, gostaria de registrar o preço de combustíveis, álcool ou diesel do posto onde eu me encontro para ele sempre ter o preço certo
    \item Eu, como desenvolvedor, desejo mapear os postos de combustíveis para que sejam tratados no sistema
    \item Eu, como motorista, gostaria de editar o meu cadastro para manter minhas informações atualizadas
\end{itemize}
Os requisitos estão sendo mantidos no Github como \textit{issues} para controlar seus status no \textit{kanban} do \citeonline{waffle} na ferramenta Waffle.io.
