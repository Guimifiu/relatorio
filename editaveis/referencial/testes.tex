\section{Testes Automatizados}

Testar software é uma atividade que consome muito tempo, mas muito importante na garantia de qualidade do produto \cite{testing-is-hard}. Os objetivos principais do teste são mostrar para o cliente e o desenvolvedor que o software implementado atende aos requisitos e descobrir falhas ou defeitos no software, apesar de não garantir a não existência delas \cite{sommerville}. Existem várias estratégias para testes; dessas estratégias derivam vários tipos de teste, tais como testes unitário, funcionais, de integração, de stress, entre outros \cite{art-of-testing}.

Uma unidade de software é o menor módulo possível de um software, geralmente sendo um método ou uma função \cite{runeson}. Segundo \citeonline{gangluo} os testes geralmente só são iniciados após uma grande parte do software já estar implementada. Para estes casos, recomenda-se o método de DTA (\textit{Direct Test Acess}), que consiste em acessar diretamente a unidade de software sem removê-la de seu contexto.

Porém, nem sempre os testes se iniciam após o desenvolvimento do software. O TDD (\textit{Test Driven Development}) é uma técnica que também faz uso de testes unitários, mas inverte a ordem normal, implementando primeiro os testes e depois as funcionalidades necessárias para a aprovação dos testes \cite{tdd}.

\citeonline{art-of-testing} define teste de integração como a verificação das interfaces das partes do sistema. Esse tipo de teste consegue identificar erros que não seriam encontrados apenas em testes unitários. Essas interfaces de diferentes partes do sistema são, por exemplo, uso de arquivos compartilhados e chamadas de métodos de outros sistemas, entre outras \cite{spillner}.


