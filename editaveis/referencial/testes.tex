\section{Testes Automatizados}

Testar software é uma atividade muito importante na garantia de qualidade do produto que consome muito tempo \cite{testing-is-hard}. Os objetivos principais de testar são mostrar para o cliente e o desenvolvedor que o software implementado atende aos requisitos e descobrir falhas ou defeitos no software, apesar de não garantir a não-existência delas \cite{sommerville}. Existem várias estratégias para testes e dessas estratégias derivam vários tipos de teste, como testes unitário, funcionais, de integração, de stress, entre outros \cite{art-of-testing}.

\subsection{Testes unitários}

Uma unidade de software é o menor módulo possível de um software, geralmente sendo um método ou uma função \cite{runeson}. Segundo \citeonline{gangluo} os testes geralmente só iniciam após uma grande parte do software já estar implementada. Para estes casos é recomendado o método de DTA, que consiste em acessar diretamente a unidade de software sem removê-la de seu contexto.

Porém, nem sempre os testes se iniciam após o desenvolvimento do software. O TDD é uma técnica que também faz uso de testes unitários mas inverte a ordem normal, implementando primeiro os testes e depois as funcionalidades que fazem os testes passarem \cite{tdd}.

\subsection{Testes de integração}

\citeonline{art-of-testing} define teste de integração como a verificação das interfaces das partes do sistema. Esse tipo de teste consegue identificar erros que, apenas através de testes unitários, não seriam encontrados. Essas interfaces de diferentes partes do sistema são, por exemplo, chamadas de métodos de outros sistemas, uso de arquivos compartilhados, entre outras \cite{spillner}.


