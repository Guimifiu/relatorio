\section{Métricas de Software}

Monitorar a qualidade de software é fundamental para qualquer projeto \cite{prmm}. De acordo com a \citeonline{iso-metricas}, métrica é uma união de procedimentos com o objetivo de definir escala e métodos para medidas. Segundo \citeonline{mills}, uma métrica é considerada boa quando ela ajuda a medir os parâmetros de qualidade de um software. 

Muitos desenvolvedores utilizam métricas para ter uma noção da completude ou não dos seus requisitos, se o design está bom e se o que foi desenvolvido foi feito com qualidade. Métricas de software também podem ser utilizadas para tomar decisões sobre o futuro de um projeto \cite{metrics-book}.

Atualmente, métricas de orientação a objetos são mais utilizadas do que outros tipos de métricas \cite{danijel}. Para o Ruby, existem algumas ferramentas como o Flog, que verifica a complexidade do código-fonte. Quanto maior a complexidade, mais difícil é o teste e a manutenção \cite{flog}. Outra ferramenta muito usada para acompanhar a qualidade do código é o Reek, que reporta os \textit{code smells} encontrados no código \cite{reek}. \textit{Code smells} mostram onde, no seu código Ruby, pode ser encontradas dificuldades de leitura, manutenção e evolução, mas não indicam se o código funciona ou não \cite{codesmells}.
