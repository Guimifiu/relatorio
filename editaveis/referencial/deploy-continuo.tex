\section{Entrega Contínua}

O processo normal de entrega de software consiste em entregar vários artefatos e realizar diversas tarefas para colocar o produto produzido em produção \cite{deploy1}. Geralmente, devido à customização necessária para implantar um software de cliente pra cliente, o processo de entrega é normalmente feito manualmente \cite{livro-deploy-continuo}.

A entrega contínua de um software é um reflexo desses problemas, bem como do primeiro princípio do manifesto ágil: \textit{"Nossa maior prioridade é satisfazer o cliente através da entrega contínua e adiantada de software com valor agregado."}\cite{manifesto}. A entrega contínua foca em \textit{build}, \textit{deploy}, teste e processo de \textit{release} \cite{livro-deploy-continuo}.
Existem cinco aspectos importantes sobre a entrega contínua \cite{network-world}:
\begin{itemize}
	\item Redução dos custos com entregas;
	\item Redução do tempo de entrega para o cliente;
	\item Redução dos riscos;
	\item Aumento da qualidade do produto;
	\item Foco na automação.
\end{itemize}
