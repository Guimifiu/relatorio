\section{Prototipação}

Segundo a \citeonline{iso-usabilidade}, usabilidade é uma medida de efetividade, eficiência e satisfação de um usuário ao utilizar um produto. Para ajudar a medir a usabilidade, além de ajudar a elicitar e validar requisitos, a prototipação é um método muito utilizado na engenharia de software \cite{sommerville}.

Protótipos variam de acordo com a fidelidade que tem ao software final. A fidelidade pode variar em interatividade, visual e conteúdo. Quanto mais elementos clicáveis, maior a proximidade do visual do protótipo ao visual final e maior a proximidade com o processo da aplicação, maior a fidelidade do protótipo \cite{nielsen}. Protótipos de alta fidelidade, que são mais interativos, fazem com que o designer não precise se preocupar tanto em fazer o protótipo funcionar e observar melhor o usuário testando, para poder tirar melhores conclusões. Em contrapartida, protótipos de baixa fidelidade são mais fáceis de serem alterados e os \textit{stakeholders} possuem uma noção melhor do quanto o trabalho está concluído \cite{nielsen}.

A técnica conhecido como \textit{Crazy Eights} é um passo do processo de \textit{design sprint} da empresa Google Ventures. O \textit{Crazy Eights} é uma técnica onde os participantes possuem 5 mimutos para desenhar as 8 telas principais do aplicativo, sendo assim 40 segundos para cada uma. O objetivo principal é ser bem direto, e como o tempo é curto, apenas os pontos cruciais para o aplicativo serão desenhados \cite{knapp}.
