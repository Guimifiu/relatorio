\section{Prototipação}

Segundo a \citeonline{iso-usabilidade}, usabilidade é uma medida de efetividade, eficiência e satisfação de um usuário ao utilizar um produto. Para ajudar a medir a usabilidade, além de ajudar a elicitar e validar requisitos, a prototipação é um método muito utilizado na engenharia de software \cite{sommerville}.
Protótipos variam de acordo com a fidelidade que tem ao software final. Protótipos de baixa fidelidade são normalmente usados para elicitar requisitos enquanto os de alta fidelidade são utilizados para validação \cite{prototipacao}.

A técnica conhecido como \textit{Crazy Eights} é um passo do processo de \textit{design sprint} da empresa Google Ventures. O \textit{Crazy Eights} é uma técnica onde os participantes possuem 5 mimutos para desenhar as 8 telas principais do aplicativo, sendo assim 40 segundos para cada uma. O objetivo principal é ser bem direto, e como o tempo é curto, apenas os pontos cruciais para o aplicativo serão desenhados \cite{knapp}.
