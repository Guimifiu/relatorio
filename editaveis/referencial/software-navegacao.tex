\section{Software de Navegação}

O sistema de navegação GPS está cada vez mais fazendo parte da vida das pessoas \cite{gps-1}. Aplicativos como o Google Maps, Waze e HERE WeGo fazem uso do sistema para poder auxiliar o usuário na decisão de rotas em seu dia-a-dia. Para auxiliar a informação \textit{online} e em tempo real das ruas, principalmente na hora de criar rotas, é comumente utilizado o RDS-TMC (Radio Data System-Traffic Message Channel) \cite{rds-tmc}. 

\subsection{Waze}
O Waze é um dos maiores aplicativos de navegação do mundo desenvolvido pela Waze Mobile de Israel e posteriormente adquirida pela Google. Sua base de dados é sustentada pelos usuários que, além de mostrar quais ruas estão engarrafadas, mostra também onde possuem \textit{blitz} da polícia, os radares e acidentes pela cidade. Além disso, é possível ver onde se localizam postos de gasolina na cidade \cite{waze}.

\subsection{Google Maps}
O Google Maps, desenvolvido pela Google, traça rotas em tempo real de acordo com informações de trânsito. Além disso, ele também permite ver as ruas com o \textit{Street View}. Além de rotas para carros, o Google Maps também mostra outras formas de se chegar nos lugares, via metrôs, ônibus e até mesmo a pé. Também é possível pela aplicação, encontrar postos de gasolina perto das rotas selecionadas \cite{google-maps}. 

\subsection{HERE WeGo}
Originalmente criado pela Nokia e antes conhecida como HERE Maps, a HERE WeGo, agora desenvolvida pela HERE Technologies, é um aplicativo que mostra rotas tanto para carros quanto para outros meios de locomoção, todas em tempo real \cite{herewego}.
