\section{Software de Navegação}

O sistema de navegação GPS está cada vez mais fazendo parte da vida das pessoas \cite{gps-1}. Aplicativos como o Google Maps, Waze e HERE WeGo fazem uso do sistema para auxiliar o usuário na decisão de rotas em seu dia a dia. Para enviar a informação \textit{online} e em tempo real das ruas, principalmente na hora de criar rotas, é comumente utilizado o RDS-TMC (\textit{Radio Data System-Traffic Message Channel}) \cite{rds-tmc}.

Um exemplo de software de navegação é o Waze, que é um dos maiores aplicativos de navegação do mundo desenvolvido pela \textit{Waze Mobile} de Israel e posteriormente adquirida pela Google. Sua base de dados é sustentada pelos usuários que, além de mostrar quais ruas estão engarrafadas, mostra também onde encontram-se \textit{blitz} da polícia, os radares e acidentes pela cidade. Além disso, é possível ver onde se localizam postos de combustíveis \cite{waze}.

Outro exemplo de software de navegação é o Google Maps, desenvolvido pela Google, traça rotas em tempo real de acordo com informações de trânsito. Além disso, ele também permite ver as ruas com o \textit{Street View}. Além de rotas para carros, o Google Maps também mostra outras formas de se chegar aos lugares, via metrô, ônibus e até mesmo a pé. Também é possível, pela aplicação, encontrar postos de combustíveis perto das rotas selecionadas \cite{google-maps}.
