\section{Software de Navegação}

O sistema de navegação GPS (\textit{Global Positioning System}) faz cada vez mais parte da vida das pessoas \cite{gps-1}. Aplicativos como Google Maps, Waze e HERE WeGo fazem uso desse sistema para auxiliar o usuário na decisão de rotas em seu dia-a-dia. Para enviar a informação \textit{online} e em tempo real das ruas, principalmente na hora de criar rotas, o RDS-TMC (\textit{Radio Data System-Traffic Message Channel}) é comumente utilizado \cite{rds-tmc}.

Um exemplo de software de navegação é o Waze, um dos maiores aplicativos de navegação do mundo desenvolvido pela \textit{Waze Mobile} de Israel e posteriormente adquirido pelo Google. Sua base de dados é sustentada pelos usuários que, além de indicar quais ruas estão engarrafadas, mostra também onde encontram-se \textit{blitz} da polícia, radares e acidentes pela cidade. Além disso, é possível ver a localização de postos de combustível \cite{waze}.

Outro exemplo de software de navegação é o Google Maps, desenvolvido pelo Google; esta aplicação traça rotas em tempo real de acordo com informações de trânsito. Além disso, ele também permite ver as ruas com o \textit{Street View}. Além de rotas para carros, o Google Maps também mostra outras formas de deslocamento, via metrô, ônibus e até mesmo a pé. Além disso, ele indica a localização de postos de combustíveis perto das rotas selecionadas \cite{google-maps}.
