\section{Integração Contínua}

Com o advento dos métodos ágeis, a entrega rápida de sofware funcional e de qualidade para o cliente teve sua importância aumentada. Uma das práticas mais importantes para atingir esse objetivo é a integração contínua \cite{continuous-integration}. 

Integração contínua é a prática onde o time integra o código periodicamente (de preferência em tempos curtos) para encontrar erros o mais rápido possível \cite{fowler}

Uma \textit{build} produzida em uma integração passa pelos seguintes passos \cite{continuous-integration}:
\begin{itemize}
\item Compilar um código-fonte;
\item Rodar uma suíte de testes; 
\item Verificar a qualidade do código via análise estática;
\item Produzir uma versão instalável de componentes pré-construídos. 
\end{itemize}

No desenvolvimento de uma ILS (\textit{Integrated Library System}), um produto multi-plataforma produzido utilizando uma adaptação descentralizada do Scrum, a integração contínua se mostrou uma das melhores práticas que contribuiu com a produtividade do trabalho, mostrando a importância da prática em projetos dessa natureza \cite{sutherland}.

