\section{Scrum}

O Scrum é um \textit{framework} de desenvolvimento ágil que consiste em um time, suas atividades associadas, artefatos e regras que servem para entregar uma série de incrementos de produto dentro de um determinado período (geralmente de 2 a 4 semanas) chamado \textit{sprint} \cite{livro-scrum}.
O time é formado por:
\begin{itemize}
\item \textbf{\textit{Product Owner}}, que é responsável por manter os requisitos e dividí-los para o resto da equipe \cite{product-owner};
\item \textbf{Scrum master}, que é o responsável por liderar e ajudar a equipe de desenvolvimento, ser a ponte entre \textit{product owner} e equipe e conduzir as reuniões para manter o status do projeto \cite{scrum-master} e;
\item \textbf{Equipe de desenvolvimento}, que são as pessoas responsáveis por desenvolver o produto e entregá-lo ao fim de cada \textit{sprint} \cite{equipe-dev};
\end{itemize}
O Scrum começa com a visão do \textit{product owner} sobre o produto e a criação das \textit{features} do sistema em uma lista priorizada chamada \textit{product backlog}. Depois, são separadas tarefas do \textit{backlog} para compor a \textit{sprint}, de acordo com o que a equipe acredita conseguir desenvolver durante o ciclo, e assim a \textit{sprint} se inicia. No final, é feita uma revisão da \textit{sprint}, onde são revisados por todos os \textit{stakeholders} os itens desenvolvidos pela equipe, e é feita uma retrospectiva, onde são mostrados os pontos positivos e negativos da \textit{sprint} a fim de melhorar a próxima \cite{livro-scrum}.
