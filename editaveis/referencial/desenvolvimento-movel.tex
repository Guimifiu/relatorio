\section{Desenvolvimento Móvel}

Atualmente, os dois sistemas operacionais para desenvolvimento móvel mais utilizados são Android e iOS \cite{gartner-top-os}. O Android possui um \textit{kernel} do Linux modificado utilizando várias bibliotecas em C/C++ \cite{mobile-dev} e o principal \textit{framework} de desenvolvimento oferecido pela Google usa o Java como linguagem de programação \cite{android}. Baseado no Mac OS X, o iOS é a plataforma da Apple para desenvolvimento móvel e usa como linguagem primária o Objective-C \cite{mobile-dev}.

Para o desenvolvimento de aplicações móveis multiplataformas, existem duas formas de implementação: a nativa, utilizando ferramentas que usam funcionalidades específicas do sistema operacional, e a híbrida, usando \textit{frameworks} que fazem o uso de HTML5, CSS3 e JavaScript \cite{mobile-dev-2}.

Aplicações nativas são específicas do sistema operacional e podem ser desenvolvidas para múltiplas plataformas com um \textit{framework} que compile um único código para todos os dispositivos \cite{hybrid-1}. A performance de aplicativos nativos é melhor pelo uso de código nativo \cite{hybrid-2}. Existem várias ferramentas para desenvolvimento multiplataforma nativo como o \textit{Xamarin}, \textit{Appecelator Titanium}, \textit{Corono} e \textit{Unity}, entre outros. Exemplos de aplicativos nativos são \textit{Angry Birds} e \textit{Shazam}.

Aplicativos híbridos são desenvolvidos utilizando ferramentas de desenvolvimento \textit{web} em junção com elementos nativos \cite{hybrid-1}. Eles modificam páginas HTML pré-empacotadas, mudando a interface de acordo com a plataforma \cite{hybrid-2}. A interface de usuário de um aplicativo híbrido possui em tempo menor de desenvolvimento de composições de natureza dinâmica \cite{mobile-dev-2}. Existem vários \textit{frameworks} para desenvolvimento híbrido, tais como o Ionic, QT e Adobe Air, dentre outros. Exemplos de aplicativos híbridos são Netflix e Linkedin.
