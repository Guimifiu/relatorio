\section{Scrum}

\subsection{Papéis}
Como a equipe de desenvolvimento é composta apenas por duas pessoas foi definido que o papel do Scrum Master não é necessário para o projeto, sendo suas atribuições divididas entre os dois membros da equipe. O papel do \textit{Product Owner} está a cargo do professor orientador.

\subsection{\textit{Sprints}}
Para o desenvolvimento da aplicação as \textit{sprints} terão duração de 2 semanas. As atividades que as compõem as \textit{sprints} são: 
\begin{itemize}
  \item \textbf{Planejamento de \textit{Sprint}}, onde são separados os itens do \textit{product backlog} que são desenvolvidas durante o período de 2 semanas;
  \item \textbf{Revisão da \textit{Sprint}}, onde são mostradas e discutidas junto ao \textit{Product Owner} as histórias de usuário que foram implementadas durante a \textit{sprint};
  \item \textbf{Retrospectiva}, onde são discutidos os pontos positivos, negativos e possíveis melhorias para as próximas iterações;
  \item \textbf{Reunião semanal}, onde são discutidas, junto ao professor orientador, os caminhos a serem seguidos no projeto;
\end{itemize}
A equipe desenvolve utilizando a estratégia do \textit{pair programming}, mas, quando a dupla não pode estar presencialmente reunida, um desenvolvedor implementa a história e o outro faz uma revisão.
