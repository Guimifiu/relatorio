\section{Scrum}

\subsection{Papéis}
Como a equipe de desenvolvimento é composta apenas por duas pessoas, foi definido que o papel so Scrum Master não é necessário para o projeto, sendo suas atribuições divididas entre os dois membros da equipe. O papel do \textit{Product Owner} ficará a cargo do professor orientador.

\subsection{\textit{Sprints}}
Para o desenvolvimento da aplicação, as \textit{sprints} terão duração de 2 semanas. As atividades que irão compor as \textit{sprints} são: 
\begin{itemize}
  \item \textbf{Planejamento de \textit{Sprint}}, onde serão separados os itens do \textit{product backlog} que serão desenvolvidas durante o período de 2 semanas;
  \item \textbf{Revisão da \textit{Sprint}}, onde serão mostradas e discutidas junto ao \textit{Product Owner} as histórias de usuário que foram implementadas durante a \textit{sprint};
  \item \textbf{Retrospectiva}, onde ser]ao discutidos os pontos positivos e negativos e possíveis melhorias para as próximas iterações;
\end{itemize}
O desenvolvimento será realizado utilizando a prática de \textit{pair programming} e, quando este não for possível, uma pessoa desenvolvendo a história e a outra revisando.
