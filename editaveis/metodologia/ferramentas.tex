\section{Ferramentas Utilizadas}

\subsection{Linguagens e \textit{Frameworks}}

O aplicativo Guimifiu está sendo desenvolvendo utilizando o framework Ionic 2, que por sua vez utiliza Typescript como linguagem de programação. O Typescript é uma linguagem que facilita o desenvolvimento em JavaScript em larga escala. É uma linguagem fortemente tipada, e que facilita a aplicação de orientação a objetos em aplicações JavaScript. O Typescript é compilado e transformado em JavaScript \cite{typescript}. !!!!!!!!!!!!FALAR DO IONIC AQUI!!!!!!!!!!!!!!!!!!!!!!

Em conjunto com o desenvolvimento do aplicativo, está sendo desenvolvido uma API RESTful utilizando a linguagem de programação Ruby com o framework Ruby on Rails. !!!!!!!!!!!!!!FALAR DO RUBY E DO RAILS AQUI!!!!!!!!!!!!!!!!. A representação dos dados da API são em JSON.

\subsection{Testes}

Para a aplicação mobile, foram utilizados os \textit{frameworks open source} Jasmine para escrever os testes e o Karma para rodar a suíte. O Karma é um ambiente de teste para Javascript onde os desenvolvedores conseguem \textit{feedbacks} rápidos do código que estão desenvolvendo \cite{karma}. O Jasmine é um \textit{framework} que utiliza conceitos de \textit{behavior-driven development} para escrever testes \cite{jasmine}. O CodeCov foi utilizado para adquirir a cobertura de código dos testes.

Para a API, foi utilizado o RSpec, que é uma ferramenta para escrever e executar a suíte de testes \cite{rspec} e o Coveralls que adquire a cobertura de código dos testes \cite{coveralls}.

\subsection{Métricas de Código-fonte}
Para as métricas de código-fonte da aplicação mobile, foi utilizado o TSLint, que checa código TypeScript em questões de manutenabilidade, leitura e erros funcionais \cite{tslint}.

No código-fonte da API, foi utilizado o CodeClimate, uma ferramenta que baixa o código do Github e checa em seus servidores a complexidade do código, duplicação, estilo e segurança \cite{codeclimate}. O CodeClimate utiliza as seguintes para gerar o relatório final:
\begin{itemize}
    \item \textbf{Reek}, para verificação de \textit{code smells}, que são sintomas no código-fonte que podem gerar problemas;
    \item \textbf{Flog}, para a verificação da complexidade do código Ruby;
    \item \textbf{Rubocop}, que verifica estilo e qualidade do código;
    \item \textbf{Brakeman}, para uma verificação estática de segurança do código;
\end{itemize}

Além disso, é também uma ferramenta própria do CodeClimate para checar duplicações de código.

O Rubocop foi utilizado também fora do CodeClimate para medir a complexidade do código. Além de complexidade, o Rubocop possui diversas outras ferramentas embutidas que verificam outros aspectos baseados no guia de estilo do Ruby mantido pela comunidade \cite{rubocop}. Os padrões mínimos ou máximos dessas outras ferramentas serão analisadas com o desenvolvimento do projeto.

\subsection{Integração e Entrega Contínua}

!!!!!!!!!!!!!!!!ALTAS TRETA!!!!!!!!!!!!!!!!!!!!!!!!!!

\subsection{Comunicação}

A comunicação da equipe foi em sua maioria pela ferramenta Slack. O Slack !!!!!!!!!!!!BLABLABLA BLABLABLA!!!!!!!!!!!!!!. As ferramentas integradas com o Slack foram:

\begin{itemize}
    \item GitHub;
    \item TravisCI;
    \item CodeCov;
    \item Coveralls;
    \item Fastlane;
\end{itemize}

Todas as ferramentas foram integradas em um único canal para facilitar a organização dos eventos durante o desenvolvimento.
