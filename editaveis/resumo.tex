\begin{resumo}

O Brasil é um dos países exportadores de petróleo com maior preço da gasolina no mundo. Em 2015, foram descobertos esquemas de cartel dos postos de combustíveis, mas, mesmo após a descoberta e tentativa de boicote pela população, os preços se mantiveram altos. O Guimifiu é um aplicativo multi-plataforma implementado com o objetivo de incentivar a população brasileira a protestar contra os preços abusivos de combustível no país. Desenvolvido em Ionic 2 com uma API em Ruby on Rails, o aplicativo está sendo desenvolvido utilizando métodos e conceitos de engenharia de software, como Scrum e entrega contínua. Apesar das dificuldades encontradas com o Ionic 2, a base do trabalho está bem preparada para a execução do processo na segunda entrega.

 \vspace{\onelineskip}

 \noindent
 \textbf{Palavras-chaves}: aplicativo. multi-plataforma. boicote. postos de combustíveis. engenharia de software. 
\end{resumo}
