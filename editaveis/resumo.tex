\begin{resumo}

Dentre os exportadores de petróleo, o Brasil é um dos países com a gasolina cara. Em 2015, foram revelados esquemas de cartel dos postos de combustíveis, mas, mesmo após a descoberta e tentativa de boicote pela população, os preços se mantiveram altos. O Guimifiu é um aplicativo multiplataforma implementado com o objetivo de incentivar a população brasileira a protestar contra os preços abusivos de combustível no país. Programado em \textit{Ionic 2} com uma API em \textit{Ruby on Rails}, o aplicativo está sendo desenvolvido utilizando métodos e conceitos da Engenharia de Software, como \textit{Scrum} e entrega contínua. As atualizações do Ionic 2 impediram que o trabalho alcançasse os resultados desejados. Apesar disso, a versão final do aplicativo está funcional e pronta para entrar em produção e ser evoluída, principalmente pelo fato da API e o módulo de administração estarem bem desenvolvidos.

 \vspace{\onelineskip}

 \noindent
 \textbf{Palavras-chaves}: Aplicativo. Multiplataforma. Boicote. Postos de combustíveis. Engenharia de Software. 
\end{resumo}
